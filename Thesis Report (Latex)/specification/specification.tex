\chapter{Related Works} \label{chap:relatedworks}

% % % % Do not delete the following % % % %
\newcommand{\usecase}[8]
{
{
\small \begin{longtable}{@{}p{.2\textwidth}p{.01\textwidth}p{.79\textwidth}@{}}
\toprule Name & & \textbf{#1} \\
\midrule Goal & & #2 \\
\midrule Actor & & #3 \\
\midrule Pre-Condition & & #4 \\
\midrule Post-Condition & & #5 \\
\midrule Post-Condition in Special Case & & #6 \\
\midrule Normal Case & \multicolumn{2}{p{.8\textwidth}}{\vspace*{-0.5cm}#7} \\
\midrule Special Cases &  \multicolumn{2}{p{.8\textwidth}}{\vspace*{-0.5cm}#8} \\
\bottomrule
\caption[Description of Use Case: #1]{Description of Use Case \term{#1}.}
\end{longtable}
}
\label{table:#1}
\clearpage
}


\usecase{Enrich Topology}
{The developer wants to enrich the application topology as per the requirements.}
{Application Developer}
{The application developer has access to the winery system and has the application requirements ready.}
{Here goes the post-condition in normal case}
{Here goes the post-condition in special case}
{\begin{enumerate}
	\item Step 1 normal case
	\item Step 2 normal case
	\item ...
\end{enumerate}}
{\begin{enumerate}
	\item[1a.] Step 1a special case
		\begin{enumerate}
			\item ...
		\end{enumerate}
	\item[2a.] Step 2a special case
		\begin{enumerate}
			\item ...
		\end{enumerate}
	\item[2b.] Step 2b special case
		\begin{enumerate}
			\item ...
		\end{enumerate}
\end{enumerate}}






